\sectionframe{Requirements}

\begin{frame}
  \frametitle{Market Research}

  \begin{block}{Features}
    \begin{itemize}
      \item Code-oriented features like syntax highlighting and line numbering
      \item Ease of sharing
      \item Plausibility of editor integration
    \end{itemize}
  \end{block}

  \begin{block}<2->{Competitors}
    \begin{itemize}
      \item<-2> \textbf{GitHub Gists}, as well as pastebin sites like
        \textbf{pastebin.com}, are code-oriented and can be shared,
        but provide no API for editor integration.

      \item<1,3> \textbf{Google Docs} can be shared, but is not
        designed for code, and has no friendly API for integration.

      \item<1,4> \textbf{yasnippet} and similar snippet tools are
        integrated with the editor (often only one editor, and can't
        be reused between editors), but provide no means of sharing.
    \end{itemize}
  \end{block}
\end{frame}

\begin{frame}
  \frametitle{Functional Requirements}

  \begin{block}<+>{Research}
    To get an idea of what users would want from such a system, if at
    all, we produced a \textbf{questionnaire}.
    % TODO What happened with the questionnaire?
  \end{block}

  \begin{block}<+>{Formulation}
    Since we used Scrum, we found it easiest to specify our functional
    requirements directly as \textbf{user stories}.  This also helped us
    verify our functional tests.
  \end{block}
\end{frame}

\begin{frame}
  \frametitle{Non-Functional Requirements}

  These were imposed on us by
  \begin{itemize}[<+>]
    \item The university, the team, and the nature of the task
    \item The possibility that we might store sensitive data
  \end{itemize}
\end{frame}